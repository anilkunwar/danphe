\documentclass[10pt,a4paper]{article}
\usepackage[utf8]{inputenc}
\usepackage[english]{babel}
\usepackage{amsmath}
\usepackage{amsfonts}
\usepackage{amssymb}
\begin{document}
\title{Finite Element Formulation of MultiSorretDiffusion Kernel}
\section{Derivation for MultiSoretDiffusion Kernel}
 Let c and T be the phase field and temperature variables respectively. For the shape function $ \phi_j$, these variables can be formulated as follow:
 \begin{equation}
 c = \sum c_j \times \phi_j
 \end{equation}
 
 \begin{equation}
 T = \sum T_j \times \phi_j
 \end{equation}
 \subsection{Residuals and Jacobian for the Kernels}
 The MultiSoretDiffusion kernel owing to the presence of the phase field variable and coupled temperature variable requires the computation of Off-diagonal Jacobian in addition to the Residual and Jacobian.
 \subsubsection{computeQpResidual}
 The residual(R) for the soret diffusion part of phase field equation that would be represented in the MOOSE kernel can be written mathematically as:
 \begin{equation}
 R = M_q \times c (1-c) \times \frac{\triangledown T}{T}
 \end{equation}
 \subsubsection{computeQpJacobian}
  The computation of Jacobian , that is computeQpJacobian in MOOSE terminology, is done by performing the derivative of residual with respect to $c_j$. Mathematically,
  \begin{equation}
  J_{c_j} = \frac{dR}{dc_j}
  \end{equation}
  \begin{displaymath}
      or, J_{c_j} = M_q \times \frac{\triangledown T}{T} \times \frac{d(c-c^2)}{dc_j} 
 \end{displaymath}
 \begin{displaymath}
 or, J_{c_j} = M_q \times \frac{\triangledown T}{T} \times [\frac{d(c_j \phi_j}{dc_j} - \frac{dc^2}{dc_j}]
  \end{displaymath} 
  Applying the chain rule for the second term in the above equation, the equation becomes
  \begin{displaymath}
    J_{c_j} = M_q \times \frac{\triangledown T}{T} \times [\phi_j \frac{d(c_j }{dc_j} - \frac{dc^2}{dc} \times \frac{dc}{dc_j}]
\end{displaymath}  
\begin{equation}
\therefore J_{c_j} = M_q \times \frac{\triangledown T}{T} \times [\phi_j  - 2c \times \phi_j] 
\end{equation} 

\subsubsection{computeQpOffDiagJacobian}
Fir the computation of off diagonal Jacobian i.e. for creating the computeQpOffDiagJacobian formula in the kernel, the residual is differentiated with respect to $T_j$. It is represented mathematically as:
\begin{equation}
  J_{T_j} = \frac{dR}{dT_j}
  \end{equation}
  \begin{displaymath}
      or, J_{T_j} = M_q \times c(1-c) \times \frac{d (\frac{\triangledown T}{T})}{dT_j} 
 \end{displaymath}
  \begin{displaymath}
      or, J_{T_j} = M_q \times c(1-c) \times [\triangledown T \frac{d (\frac{1}{T})}{dT_j} + \frac{1}{T} \times \frac{d\triangledown T}{dT_j}]
 \end{displaymath}
 \begin{displaymath}
      or, J_{T_j} = M_q \times c(1-c) \times [\triangledown T \frac{dT^{-1}}{dT} \frac{dT}{dT_j} + \frac{1}{T} \times \bigtriangledown (\frac{dT}{dT_j})]
 \end{displaymath}
 \begin{displaymath}
 or, J_{T_j} = M_q \times c(1-c) \times [-\frac{\triangledown T}{T^2} \times \phi_j + \frac{\triangledown \phi_j}{T}]
 \end{displaymath}
 
 \begin{equation}
 \therefore J_{T_j} = M_q \times c(1-c) \times [\frac{\triangledown \phi_j}{T} - \frac{\triangledown T}{T^2} \times \phi_j]
 \end{equation}
 
  
\end{document}
